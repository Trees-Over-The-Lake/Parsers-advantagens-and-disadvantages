\documentclass{article}

\usepackage[
backend=biber,
sorting=ynt
]{biblatex}
\addbibresource{reference.bib}

\title{Usage of LALR and LR(1) for Synthetic Code Generation}
\author{Lucas Santiago de Oliveira}
\date{September 25, 2023}

\begin{document}
    \maketitle

    It's pretty hard to find a fast and efficient LR(1) parser. That's the first idea that
    the article authors wants to make clear. We need to keep it in mind: creating a LALR and LR(1) can be extremely
    difficult and takes high time and space complexy. This type of parser easily goes up to thousands and thousands of automata states, that
    means no one can write all the states by hand, needing to create an algorithm to generate automatically.

    If it is so hard to create those types of parser why do the authors wants to do that? First, at the time of
    creation of this article\cite{hyacc}, there weren't many LALR and LR(1) parsers available. So,
    the authors decided to create one test and make the code open source so everyone could test it and use it.
    Other important factors are error handling is much more robust than a simple LL parser and after built the efficiency
    is superior, parsing code much faster than LL.

    % Eficciency, code parsing speed - article tests
    The main goal of the article is to show the efficiancy of their parser. When they compared to other
    implementations they got amazing performance for both cases.

    \begin{figure}[h!]
        \centering
        \begin{tabular}{|c|c|c|c|}
            \hline
            Parsers & Worst Time & Best Time & Best implementation \\
            \hline
            Menhir & 1.97 & 1.48 & PG MLR(1) \\
            \hline
            MSTA & 5.32 & 1.17 & LALR(1) \\
            \hline
            Hyacc & 3.53 & 1.10 & LALR(1) \\
            \hline
        \end{tabular}
        \caption{Time for parsing a C++ grammar in seconds for different parsers.}
    \end{figure}

    \begin{figure}[h!]
        \centering
        \begin{tabular}{|c|c|c|c|}
            \hline
            Parsers & Worst Time & Best Time & Best implementation \\
            \hline
            Menhir & 1.64 & 0.56 & PG MLR(1) \\
            \hline
            MSTA & 0.92 & 0.13 & LALR(1) \\
            \hline
            Hyacc & 1.05 & 0.19 & LALR(1) \\
            \hline
        \end{tabular}
        \caption{Time for parsing a C grammar in seconds for different parsers.}
    \end{figure}

    Looking to those two figures it is noticiable that in both cases LALR(1) was the clear winner
    being the fastest parser. For C++ Hyacc got the best time of those three parsers, for C it got the
    second position 0.06 seconds behind the first place [MSTA] that used LALR(1) as well.

    % Paragraph about error handling
    In general LR(k) parsers have really good error handlings. This type of parser is capable of detecting missing tokens
    adding them where they are missing, so the parser doesn't need to stop before parsing the rest of the file
    giving more information in just one compilation try. Hyacc specifically doesn't give really good
    error messages, just like said in the section 2.6 of Hyacc article\cite{hyacc}. Considering that the project
    is open source anyone can just fix this making this program even better.

    % Conclusion
    \Citeauthor{hyacc} tried to create a new and document scassed type of parser LR(1).
    Yacc and Bison are two very famous LALR(1) parsers that were already wide used by the community.
    Considering the superior performance, it's uniqueness, coverage, efficiency, portability,
    usability and availability of Hyacc, this algorithm was very important to present to the community
    a new important algorithm to parse grammars.

    % Show all references used
    \printbibliography
\end{document}
